\documentclass[12pt]{article}
\usepackage[leqno]{amsmath}
\usepackage{amsfonts,amsmath,amssymb}
\usepackage{enumitem}
\usepackage[width=6.5in,height=10in]{geometry}
\parindent=0pt
\parskip=12pt
\newcommand\lat[1]{{#1}_{\rm lat}}
\renewcommand\long[1]{{#1}_{\rm long}}
\begin{document}
\centerline{\Large\bf The Geosphere}
\section{Introduction}

This note describes two elementary computations in spherical geometry which have
practical value for navigation on the surface of the earth.

To describe the computations we need a few conventions. We will consider the
unit sphere $\Sigma$ in ${\mathbb R}^3$ centered at the origin. A {\it great
  circle} is the intersection of $\Sigma$ with a plane through the origin. The
{\it equator} is the great circle contained in the $x$-$y$ plane. A {\it geodesic path}
is a unit-speed parametrized arc of a great circle. A geodesic path has an
orientation determined by its parametrization and hence has a starting point and
an ending point.

Angles are measured in radians and have values in the interval $[0, 2\pi)$. If
$\gamma$ and $\gamma'$ are two geodesic paths starting at the same point $S$
then the {\it angle} from $\gamma$ to $\gamma'$ is the angle from the velocity
vector of $\gamma$ at $S$ to the velocity vector of $\gamma'$ at $S$, measured in
the clockwise direction as viewed from outside $\Sigma$.

The {\it north pole} of $\Sigma$ is the point $(0, 0, 1)$ and the {\it south
  pole} is the point $(0, 0, -1)$. A {\it meridian} is a geodesic path from the
south pole to the north pole. There is a unique meridian passing through any
non-polar point $S$ of $\Sigma$, denoted $\mathcal{M}_s$. The {\it bearing} of a
geodesic path $\gamma$ which starts at a non-polar point $S$ is the angle from
$\mathcal{M}_s$ to $\gamma$. The {\it prime meridian} $\mathcal P$ is the
meridian through the point $(1, 0, 0)$.

The (unit) position vector of a point $P$ on $\Sigma$ will be denoted
$\vec{P}$.  The distance between two points $p$ and $q$ of $\Sigma$ is
equal to the (smaller) angle between the vectors $\vec{P}$ and
$\vec{Q}$, so the dot product $\vec{P}\cdot \vec{Q}$ is the cosine of
the distance between the points.  When dealing with longitudes, which
may not lie in the interval $[0, \pi]$, we will have
to deal with the fact that $\arccos(\cos \theta) = -\theta$ when
$\theta\in [-\pi, 0]$.

The two computations are:
\begin{enumerate}[label={(\arabic*)}]
\item Given coordinates of a non-polar point $S$ on $\Sigma$, an angle
  $\beta$, and a distance $\delta$, find the coordinates of the ending point of the
  geodesic path starting at $S$ with bearing $\beta$ and length $\delta$.
\item Given coordinates of a non-polar point $S$ and of a different point
  $E$ which is not the antipode of $S$, find the bearing and length of
  the (shorter) geodesic path from $S$ to $E$.
\end{enumerate}
If $S$ is a pole then computation (2) is trivial since the bearing is $0$
if $S$ is the south pole or $\pi$ if $S$ is the north pole and the
distance is $\pi - |\lat S|$.

\section{Coordinates}
The {\it latitude} $\lat s$ of a point $s\in\Sigma$ is the signed distance from $S$ to
the point where $\mathcal{M}_s$ meets the equator, the sign being the same
as the sign of the $z$-coordinate of $S$. So latitudes take values in the
interval $[-\pi, \pi]$.

The {\it longitude} $\long s$ of a point$s\in\Sigma$ is the angle between
$\mathcal{P}$ and $\mathcal{M}_s$.  Longitude is positive for points to the
east of the prime meridian and negative for points to the west, and longitudes
take values in the interval $(-\pi, \pi]$.

The {\it coordinates} of $S$ refers to the pair $(\lat s, \long s)$.  

\section{Symmetry}

Latitude, bearing and distance are all preserved by rotation about the $z$-axis,
so we may assume for both of these problems that the point $S$ lies on the prime
meridian.

For computation (1) we will replace $S$ by the point $S'$ with coordinates
$(\lat s, 0)$. If the computation for $S'$ produces a point with longitude
$\lambda$ then the result for the initial problem would be a point with
longitude $-\pi + (\pi + v + \lambda \mod 2\pi)$. The latitude of the result
will be the same for $S$ and $S'$, so no adjustment is needed for the latitude.

For computation (2) we will replace the point $S$ by the point with coordinates
$(\lat S, 0)$ and the point $E$ by the point with coordinates
$(\lat E, -\pi + (\pi + \long E - \long S \mod 2\pi)$.  No adjustment is
needed for the result.

\section{Bearing and Distance to Coordinates}
Here we consider computation (1).  We are given a starting point $S$ with coordinates
$(\lat s, 0)$ a bearing $\beta$, and a distance $\delta$.  We need to compute
the coordinates $(\lat E, \long E)$ of the ending point $E$ of the geodesic path
$\gamma$ with bearing $\beta$ and length $\delta$.

First we have
\[
  \vec{S} = (\cos\lat S, 0, \sin\lat S)
\]
and
\[
  \vec{E} = (\cos\long E\cos\lat E, \sin\long E\cos\lat E, \sin\lat E) .
\]
Thus
\[
  \cos \delta = \vec{S}\cdot\vec{E} = \cos\lat S \cos\long E\cos\lat E + \sin\lat S \sin\lat E .
  \eqno{\ast}
\]

Second, observe the angle between the prime meridian and the path $\gamma$ equals
the angle between the normal vectors to the planes cutting out their great
circles.  These normal vectors are $\vec N = (0, -1, 0)$ for the meridian and
$\frac{1}{\sin\delta}\vec{S}\times\vec{E}$ for $\gamma$.  The dot product
$\vec{S}\times\vec{E}\cdot\vec{N}$ is given by
\[
  \det\left[
    \begin{matrix}
      \cos\lat S & 0 & \sin\lat S\\
      \cos\long E\cos\lat E& \sin\long E\cos\lat E & \sin\lat E\\
      0 & -1 & 0\\
    \end{matrix}
    \right]\leqno{(\ast)}
\]
so we have
 \[
 \sin\delta\cos\beta = \cos\lat S\sin\lat E - \sin\lat S\cos\long E\cos\lat E .
 \leqno{(\dagger)}
\]

Taking $(\ast)$ and $(\dagger)$ together gives a $2\times2$ linear system that
determines $\sin\lat E$ and $\cos\long E\cos\lat E$ in terms of  $S$, $\beta$
and $\delta$:
\[
  \left[
  \begin{matrix}
   \cos\lat S  & \sin\lat S \\
   -\sin\lat S & \cos\lat S
  \end{matrix}
  \right]
  \left[
   \begin{matrix}
    \cos\long E\cos\lat E \\
    \sin\lat E
   \end{matrix}
  \right]
  = \left[
    \begin{matrix}
    \cos\delta \\
    \sin\delta\cos\beta \\
    \end{matrix}
  \right]
\]

Solving the system, by multiplying by the rotation matrix with angle $-\lat S$,
gives
\begin{align*}
  \cos\long E\cos\lat E &= \cos\lat S\cos\delta -
                          \sin\lat S\sin\delta\cos\beta\\
  \sin\lat E &= \sin\lat S\cos\delta + \cos\lat S\sin\delta\cos\beta
\end{align*}

We may then solve for $\lat E$ and $\long E$ in two steps.
\begin{align*}
  \lat E &= \arcsin(\sin\lat S\cos\delta + \cos\lat S\sin\delta\cos\beta)\\
  \long E &= \pm\arccos((\cos\lat S\cos\delta -
            \sin\lat S\sin\delta\cos\beta) / \cos\lat E)
\end{align*}

The ambiguity in the sign of $\long E$ arises because the longitude of
$E$ may have any value in $(-\pi, \pi)$ but
$\arccos(\cos\theta) = -\theta$ for $\theta\in(-\pi,0)$.  The
longitude changes monotonically along any great circle which does not
pass through a pole, and is constant along great circles which do pass
through a pole.  So we know that the sign must be positive if the
bearing lies in $(0, \pi)$ and negative if the bearing lies in
$(\pi, 2\pi)$.  The sign is irrelevant if the bearing is $0$ or $\pi$,
since the longitude will be $0$ at each point of the great circle.

\section{Coordinates to Bearing and Distance}
Now we consider computation (2).

\end{document}.
